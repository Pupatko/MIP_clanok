\documentclass{article}

% Language setting
% Replace `english' with e.g. `spanish' to change the document language
\usepackage[slovak]{babel}

% Set page size and margins
% Replace `letterpaper' with`a4paper' for UK/EU standard size
\usepackage[letterpaper,top=2cm,bottom=2cm,left=3cm,right=3cm,marginparwidth=1.75cm]{geometry}

% Useful packages
\usepackage{amsmath}
\usepackage{graphicx}
\usepackage[colorlinks=true, allcolors=blue]{hyperref}

\title{Vplyv umelej inteligencie na vyhľadávacie systémi}
\author{Damián Tancoš}

\begin{document}
\maketitle

\section{Spresnenie témy}

Táto téma sa zaoberá hlbokým vplyvom umelej inteligencie na funkčnosť a výkonnosť vyhľadávačov. Skúma, ako algoritmy umelej inteligencie, spracovanie prirodzeného jazyka a techniky strojového učenia menia podobu vyhľadávania informácií. Od personalizovaných výsledkov vyhľadávania až po rozpoznávanie hlasu. Pokroky poháňané umelou inteligenciou revolučne menia skúsenosti používateľov a formujú budúcnosť vyhľadávania informácií online. Tento cielený prieskum umožňuje hĺbkovú analýzu toho, ako umelá inteligencia mení tradičné vyhľadávače, čím sa stávajú efektívnejšími a schopnými pochopiť zámery používateľov na hlbokej úrovni.

\subsection{Personalizácia:}
 Vyhľadávače poháňané umelou inteligenciou sa prispôsobujú individuálnym preferenciám a ponúkajú personalizované výsledky vyhľadávania a odporúčania prispôsobené každému používateľovi.

\subsection{Rozpoznávanie hlasu a obrazu:}
 Umožňuje rozpoznávanie hlasu a obrazu, vďaka čomu môžu používatelia vyhľadávať pomocou prirodzeného jazyka alebo vizuálnych podnetov, čím sa vyhľadávače stávajú intuitívnejšími a prístupnejšími.

\subsection{Sémantické porozumenie:}
Umelá inteligencia zlepšuje sémantické porozumenie vyhľadávacích dotazov, čím umožňuje kontextovo relevantnejšie výsledky a zvyšuje presnosť vyhľadávania informácií.

\subsection{Sumarizácia obsahu:}
 Algoritmy AI dokážu zhrnúť dlhé články alebo dokumenty a poskytnúť používateľom stručné a ľahko stráviteľné informácie.
Prediktívne vyhľadávanie: predpovedá zámery používateľa a ponúka návrhy ešte pred úplným zadaním dotazu, čím urýchľuje proces vyhľadávania.

\subsection{Vyhľadávanie v prirodzenom jazyku:}
 Umožňuje používateľom komunikovať s vyhľadávačmi pomocou konverzačného jazyka, čím sa preklenuje priepasť medzi ľudským jazykom a digitálnymi informáciami.

\subsection{Filtrovanie falošných správ:}
Môže pomôcť pri identifikácii a filtrovaní dezinformácií a falošných správ z výsledkov vyhľadávania, čím sa zvýši spoľahlivosť informácií.

\section{Zdroje}

\subsection{Pdfka:}
\url{https://citeseerx.ist.psu.edu/pdf/e631be2638577c34fbfce510fe2a17a48d7a11e4} \\
\url{https://citeseerx.ist.psu.edu/pdf/8111117e38eca27207d7bcb39029202c6bcfdc32} \\
\url{https://openai.com/blog} \\

\subsection{Random:}
\url{https://www.youtube.com/watch?v=kzeJg3fk7dw} \\
\url{https://blog.google/products/search/generative-ai-search/} \\

\end{document}